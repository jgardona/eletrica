\documentclass{article}
\usepackage{tikz, tkz-euclide}
\usepackage[siunitx]{circuitikz}
\usepackage[utf8]{inputenc}
\usepackage[portuguese]{babel}
\usepackage{amsmath}
\usepackage{graphicx}
\usepackage[hidelinks]{hyperref}
\usepackage{float}
\usepackage{svg}
\usepackage{siunitx}

\title{Teorema de Thévenin}
\author{Julio C. B. Gardona}
\date{\today}

\begin{document}
	\maketitle
	\begin{abstract}
		Este documento tem por objetivo servir como estudo de caso sobre o Teorema de Thévenin e prover alguns exercícios de simplificação de circuitos com fontes dependentes e independentes. 
	\end{abstract}
	
	\section{Introdução}
	
	O \textbf{teorema de Thévenin} afirma que u\textit{m circuito linear com dois terminais} pode ser substituído por um circuito equivalente formado por uma fonte de tensão $V_{th}$ em série com um resistor $R_{th}$, onde $V_{th}$ é a tensão de circuito aberto nos terminais, e $R_{th}$ é a resistência de entrada ou equivalente nos terminais quando as fontes independentes forem desativadas.
	
	O teorema é amplamente utilizado onde a carga $R_l$ é variável, obrigando o circuito a ser recalculado sempre que a carga for modificada.
	
	\begin{figure}[H]
		\centering
		\begin{circuitikz}[american, thick]
			\coordinate (a) at (0, 0);
			\coordinate (b) at (3, 3);
			\coordinate (c) at (3, 0);
			\coordinate (d) at (6, 3);
			\coordinate (e) at (6, 0);
			\coordinate (f) at (7, 3);
			\coordinate (g) at (7, 0);
			\coordinate (h) at (10, 3);
			
			% Draw first box
			\draw (a) rectangle (b) node[inner sep=5px, midway, text width=50px, align=center]{Circuito linear com dois terminais};
			\draw (b) to[short, -o, f=$I$] (d) node[above]{a} to[short] (f);
			\draw (c) to[short, -o] (e) node[below]{b} to[short] (g);
			% Draw second box
			\draw (g) rectangle (h) node[inner sep=5px, midway, text width=50px, align=center]{Carga};
			\draw (d) to[open, v=$V$] (e);
		\end{circuitikz}
		\caption{Circuito Original}
		\label{fig:ex1}
	\end{figure}
	
	\begin{figure}[H]
		\centering
		% \includesvg[width=0.8\textwidth, inkscapelatex=false]{svg/ceqthevenin}
		\begin{circuitikz}[american, thick]
			\coordinate (gnd) at (0, 0);
			\coordinate (a_rth) at (0, 3);
			\coordinate (b_rth) at (3, 3);
			\coordinate (a_uonode) at (4, 3);
			\coordinate (b_uonode) at (5, 3);
			\coordinate (a_donode) at (4, 0);
			\coordinate (b_donode) at (5, 0);
			\coordinate (endcarga) at (8, 0);

			
			\draw (gnd) to[vsource=$V_{th}$, invert] (a_rth)
			to[R=$R_{th}$] (b_rth)
			to[short] (a_uonode)
			to[short] (b_uonode);
			\draw (a_donode) to[open, o-o, v=$V$] (a_uonode); 
			\draw (a_donode) to[short] (gnd);
			\draw (a_donode) to[short] (b_donode);
			\draw (b_uonode) rectangle (endcarga) node[midway, align=center]{Carga};
		\end{circuitikz}
		\caption{Circuito Equivalente de Thévenin}
		\label{fig:ex2}
	\end{figure}
	
	\section{Estudo com Fontes Dependentes e Independentes}
	
	Para encontrarmos $R_{th}$, os terminais $a$ e $b$ devem ser desconectados, dessa forma, nenhuma corrente fluirá por eles. Precisamos desligar todas as fontes de tensão e fontes de corrente independentes. As fontes dependentes necessitam de variáveis do circuito e não podem ser desligadas. Uma fonte de tensão \textbf{desligada} significa ser trocada por um \textit{curto circuito}, enquanto uma fonte de corrente \textbf{desligada} significa ser trocada por um \textit{circuito aberto}. A resistência equivalente desse circuito  medida nos terminais deve ser igual a $R_{th}$, ou seja,
	
	$$
	R_{th} = R_{oc}
	$$
	
	A tensão medida nos terminais $a$ e $b$, com a \textbf{carga desconectada} e suas \textbf{fontes ativadas}, deverão ser iguais a $V_{th}$, ou seja,
	
	$$
	V_{th} = V_{oc}
	$$
	
	O teorema de Thévenin é muito importante na análise de circuitos, porque ajuda a simplificar circuitos complexos, e um circuito complexo pode ser substituído por uma fonte de tensão independente e um único resistor.
	
	Consideraremos um circuito linear terminado por uma carga $R_l$ conforme mostra a figura 3. A corrente $I_l$ através da carga e a tensão $V_l$ na carga são facilmente determinadas, uma vez que seja obtido o circuito equivalente.
	
	\subsection{Problema Prático 4.8}
	Usando o teorema de Thévenin, determine o circuito equivalente à esquerda dos terminais do circuito da figura \ref{fig:ex4}. Em seguida determine $I$.
	
	\begin{figure}[H]
		\centering
		\begin{circuitikz}[american, thick]
			\coordinate (gnd) at (0, 0);
			\coordinate (a) at (0, 3);
			\coordinate (b) at (3, 3);
			\coordinate (c) at (6, 3);
			\coordinate (d) at (8, 3);
			\coordinate (e) at (10, 3);
			\coordinate (f) at (10, 0);
			\coordinate (g) at (8, 0);
			\coordinate (h) at (6, 0);
			\coordinate (i) at (3, 0);
			
			% Implementar dispositivos em paralelo
			
			\draw (gnd) to [vsource=12<\volt>, invert] (a);
			\draw (i) to [isource, l=2<\ampere>] (b);
			\draw (h) to [R=4<\ohm>] (c);
			\draw (g) node[above]{b} to [open, o-o] (d) node[above]{a};
			\draw (f) to [R=1<\ohm>, f<=$I$] (e);
			
			% Implementar dispositivos em série
			\draw (a) to [R=6<\ohm>] (b)
			to [R=6<\ohm>] (c)
			to [short] (d)
			to [short] (e);
			\draw (f) to [short] (g)
			to [short] (h)
			to [short] (i)
			to [short] (gnd);
		\end{circuitikz}
		\caption{Esquema para problema prático 4.8 }
		\label{fig:ex4}
	\end{figure}
	
	Determinamos $R_{th}$ desativando a fonte de $12V$ (substituindo-a por um curto circuito) e a fonte de corrente de $2A$ (substituindo-a por um circuito aberto). Terminamos com dois resistores de $6~\Omega$ em série, e um de $4~\Omega$ em paralelo. O circuito torna-se aquele mostrado na figura \ref{fig:schfontoff} .
	
	\begin{equation}
		\centering
		\begin{aligned}
			(6~\Omega + 6~\Omega) \parallel 4~\Omega &= 3~\Omega \\
			R_{th} &= 3~\Omega
		\end{aligned}
	\end{equation}
	
	\begin{figure}[H]
		\centering
		\begin{circuitikz}[american, thick]
			\coordinate (gnd) at (0, 0);
			\coordinate (a) at (0, 3);
			\coordinate (b) at (3, 3);
			\coordinate (c) at (6, 3);
			\coordinate (d) at (7.5, 3);
			\coordinate (e) at (7.5, 0);
			\coordinate (f) at (6, 0);
			
			% Implementar dispositivos em paralelo
			\draw (gnd) to[short] (a);
			\draw (c) to[R=4<\ohm>] (f);
			\draw (d) to[open, o-o, l=$R_{th}$] (e);
			
			% Implementar dispositivos em série
			\draw (a) to[R=6<\ohm>] (b) to[R=6<\ohm>] (c);
			\draw (c) to[short, *-] (d);
			\draw (e) to[short] (f) to[short] (gnd);
		\end{circuitikz}
		\caption{Circuito com as fontes desativadas.}
		\label{fig:schfontoff}
	\end{figure}
	
	Para determinar $V_{th}$ consideraremos o circuito da figura \ref{fig:vth01}. Aplicando análise nodal, obtemos
	
	\begin{equation}
		\centering
		\begin{aligned}
			i_1 + 2~A &= i_2 \\
			\frac{12 - V}{6} + 2 &= \frac{V}{6 + 4} = 15~V
		\end{aligned}
	\end{equation}
	
	Encontramos o nó $V=15~V$. O nó de $V_{th}$ está no divisor de tensão, ou seja, acima do resistor de $4~\Omega$. Logo
	
	\begin{equation}
		\centering
		\begin{aligned}
			V_{th} = 15 \cdot \frac{4}{6 + 4} = 6~V
		\end{aligned}
	\end{equation}
	
	A corrente $I$ é simplesmente $I = \frac{V_{th}}{R_{th} + 1} = 1.5~A$. Esse cálculo é baseado no circuito equivalente da figura \ref{fig:equivalente1}.
	
	\begin{figure}[H]
		\centering
		\begin{circuitikz}[american, thick]
			\coordinate (vsource_a) at (0, 0);
			\coordinate (vsource_b) at (0, 3);
			\coordinate (isource_u) at (3, 3);
			\coordinate (isource_d) at (3, 0);
			\coordinate (r4o_u) at (6, 3);
			\coordinate (r4o_d) at (6, 0);
			\coordinate (oc_u) at (7.5, 3);
			\coordinate (oc_d) at (7.5, 0);
			
			\draw (vsource_a) to [vsource=12<\volt>, invert] (vsource_b);
			\draw (isource_d) to [isource, l=$2A$, -*] (isource_u) node[above]{V};
			\draw (r4o_u) to [R=4<\ohm>] (r4o_d);
			\draw (oc_u) to[open, o-o, v=$V_{th}$] (oc_d);
			\draw (vsource_b) to[R=6<\ohm>] (isource_u);
			\draw (isource_u) to[R=6<\ohm>] (r4o_u);
			\draw (vsource_a) to[short] (isource_d);
			\draw (isource_d) to[short] (r4o_d) to[short] (oc_d);
			\draw (r4o_u) to[short] (oc_u);
		\end{circuitikz}
		\caption{Determinando $V_{th}$.}
		\label{fig:vth01}
	\end{figure}
	
		\begin{figure}[H]
		\centering
		\begin{circuitikz}[american, thick]
			\coordinate (vsource-u) at (0, 3);
			\coordinate (vsource-d) at (0, 0);
			\coordinate (r1o-u) at (3, 3);
			\coordinate (r1o-d) at (3, 0);
			
			\draw (vsource-d) to[vsource=6<\volt>, invert] (vsource-u);
			\draw (vsource-d) to[short] (r1o-d) to[R=1<\ohm>] (r1o-u);
			\draw (vsource-u) to[R=3<\ohm>] (r1o-u);
		\end{circuitikz}
		\caption{Circuito equivalente.}
		\label{fig:equivalente1}
	\end{figure}
	
	\subsection{Problema Prático 4.9}
	Determine o circuito equivalente de Thévenin da figura \ref{fig:pp4.9.1}, à esquerda dos terminais $a$ e $b$.
	
	Esse circuito contém uma fonte dependente. Diferentemente do anterior, não podemos desligar todas as fontes. Nesse caso desligamos somente a fonte de tensão e excitamos a rede com uma fonte de tensão $v_0=1V$ conectado aos terminais $a$ e $b$ para podermos encontrar $R_{th}$. Logo, nossa análise se torna mais complicada pois precisaremos de mais processos para resolver o circuito equivalente. Abaixo, seguiremos passo a passo. 

	
	\subsubsection{Encontrando $V_{th}$}
	
	Para encontrar $V_{th}$, utilizaremos o circuito da figura \ref{fig:pp4.9.1}. Iremos utilizar \textbf{LKC} para encontrar a tensão no nó $p$, e toda configuração da rede, como a direção das correntes e os nós estão de acordo com a figura \ref{fig:pp4.9.2}.
	\begin{figure}[H]
		\centering
		\begin{circuitikz}[american, thick]
			\coordinate (vsource-down) at (0, 0);
			\coordinate (vsource-up) at (0, 3);
			\coordinate (disource-down) at (3, 0);
			\coordinate (disource-up) at (3, 3);
			\coordinate (r4p-up) at (6, 3);
			\coordinate (r4p-down) at (6, 0);
			\coordinate (open-up) at (9, 3);
			\coordinate (open-down) at (9, 0);
			
			\draw (vsource-up) to[vsource=6<\volt>] (vsource-down);
			\draw (disource-down) to[cisource, l_=$1.5 \cdot I_x$] (disource-up);
			\draw (r4p-up) to[R=4<\ohm>] (r4p-down);
			\draw (open-up) node[right]{a} to[open, o-o] (open-down) node[right]{b};
			\draw (vsource-up) to[R=5<\ohm>] (disource-up);
			\draw (disource-up) to[R=3<\ohm>, f>^=$I_x$] (r4p-up);
			\draw (r4p-up) to[short] (open-up);
			\draw (r4p-down) to[short] (open-down);
			\draw(vsource-down) to[short] (disource-down) to[short] (r4p-down);
			
		\end{circuitikz}
		\caption{Esquema para o problema prático 4.9}
		\label{fig:pp4.9.1}
	\end{figure}
	
	Temos duas correntes, $I_1$ e $1.5 I_x$ entrando no nó $p$. $I_x$ sai do nó. De acordo com a \textit{LKC}, $\sum I_{in} = \sum I_{out}$.
	
	\begin{equation}
		\centering
		\begin{aligned}
			I_1 + 1.5 \cdot I_x &= I_x \\
			\frac{6-V_p}{5} + 1.5 \cdot I_x &= \frac{V_p}{3+4} \\
			V_p &\approx \SI{9.333}{\volt}
		\end{aligned}
	\end{equation}
	
	Lembrando, $I_x = \frac{V_p}{3+4}$. Nesse caso, iremos encontrar $V_{th}$ no divisor de tensão(entre os resistores de $\SI{3}{\ohm}$ e $\SI{4}{\ohm}$), no nó $a$.
	
	\begin{equation}
		\centering
		\begin{aligned}
			V_{th} &= V_p \cdot \frac{4}{4+3} \\
			V_{th} &\approx \SI{5.333}{\ohm} 
		\end{aligned}
	\end{equation}
	
	\begin{figure}[H]
		\centering
			\begin{circuitikz}[american, thick]
				\coordinate (vsource-down) at (0, 0);
				\coordinate (vsource-up) at (0, 3);
				\coordinate (disource-down) at (3, 0);
				\coordinate (disource-up) at (3, 3);
				\coordinate (r4p-up) at (6, 3);
				\coordinate (r4p-down) at (6, 0);
				\coordinate (open-up) at (9, 3);
				\coordinate (open-down) at (9, 0);
				
				\draw (vsource-up) to[vsource=6<\volt>] (vsource-down);
				\draw (disource-down) to[cisource, l_=$1.5 \cdot I_x$, -*] (disource-up) node[above]{p};
				\draw (r4p-up) to[R=4<\ohm>] (r4p-down);
				\draw (open-up) node[right]{a} to[open, o-o] (open-down) node[right]{b};
				\draw (vsource-up) to[R=5<\ohm>, f>^=$I_1$] (disource-up);
				\draw (disource-up) to[R=3<\ohm>, f>=$I_x$] (r4p-up);
				\draw (r4p-up) to[short] (open-up);
				\draw (r4p-down) to[short] (open-down);
				\draw(vsource-down) to[short] (disource-down) to[short] (r4p-down);
				
			\end{circuitikz}
		\caption{Esquema preparado para análise nodal}
		\label{fig:pp4.9.2}
	\end{figure}
	
	\subsubsection{Encontrando $R_{th}$}
	
	Para encontrar $R_{th}$, precisaremos desligar a fonte de tensão de \SI{6}{\volt} e excitar a rede inserindo uma tensão(que pode ser da nossa escolha) entre os terminais $a$ e $b$. Conforme já foi mencionado acima, não podemos desligar fontes dependentes, pois essas dependem de variáveis de circuito.
	
	\begin{figure}[H]
		\centering
		\begin{circuitikz}[american, thick]
			\coordinate (vsource-down) at (0, 0);
			\coordinate (vsource-up) at (0, 3);
			\coordinate (disource-down) at (3, 0);
			\coordinate (disource-up) at (3, 3);
			\coordinate (r4p-up) at (6, 3);
			\coordinate (r4p-down) at (6, 0);
			\coordinate (open-up) at (8.5, 3);
			\coordinate (open-down) at (8.5, 0);
			
			\draw (vsource-up) to[short] (vsource-down);
			\draw (disource-down) to[cisource, l_=$1.5 \cdot I_x$, -*] (disource-up) node[above]{p};
			\draw (r4p-up) node[above, xshift=5pt]{$v_a$} to[R=4<\ohm>, f>^=$I_2$, *-] (r4p-down);
			\draw (open-up) node[right]{a} to[open, o-o] (open-down) node[right]{b};
			\draw (vsource-up) to[R=5<\ohm>, f<^=$I_1$] (disource-up);
			\draw (disource-up) to[R=3<\ohm>, f>=$I_x$] (r4p-up);
			\draw (r4p-up) to[short, f<^=$I_0$] (open-up);
			\draw (r4p-down) to[short] (open-down);
			\draw(vsource-down) to[short] (disource-down) to[short] (r4p-down);
			\draw (open-up) to[vsource=1<\volt>] (open-down);
		\end{circuitikz}
		\caption{Encontrando $R_{th}$}
		\label{fig:pp4.9.3}
	\end{figure}
	
	O objetivo é encontrar a corrente $I_0$, e usá-la para calcular $R_{th}$.
	
	\begin{equation}
		\begin{aligned}
			R_{th} = \frac{v_a}{I_0} &= \frac{1}{I_0} \\
			1.5 \cdot I_x &= I_1 + I_x \\
			1.5 \cdot \frac{v_p - v_a}{3} &= \frac{v_p}{5} + \frac{v_p - v_a}{3} \\
			1.5 \cdot \frac{v_p - 1}{3} &= \frac{v_p}{5} + \frac{v_p - 1}{3} \\
			v_p &\approx \SI{-5.0}{\volt} \\
			I_x + I_0 &= I_2 \\
			\frac{v_p - 1}{3} + I_0 &= \frac{1}{4} \\
			\frac{-5 - 1}{3} + I_0 &= \frac{1}{4} \\
			I_0 &\approx \SI{2.250}{\ampere} \\
			R_{th} &= \frac{v_a}{I_0} \\
			R_{th} &= \frac{1}{2.250} \approx \SI{0.444}{\ohm}
		\end{aligned}
	\end{equation}

	\subsection{Problema Prático 4.10}

	Encontre o equivalente de Thévenin do circuito da figura \ref{fig:pp4.10}.
	
	\subsubsection{Definição}
	
	O primeiro fato a ser considerado é que, uma vez que não há nenhuma fonte independente no circuito, temos de excitá-lo externamente. Quando não há nenhuma fonte independente, não teremos um valor para $V_{th}$. Podemos encontrar apenas $R_{th}$.

	\begin{figure}[H]
		\centering
		\begin{circuitikz}[american, thick]
			\coordinate (vx-down) at (0, 0);
			\coordinate (vx-up) at (0, 3);
			\coordinate (r15-up) at (6, 3);
			\coordinate (r15-down) at (6, 0);
			\coordinate (r10b) at (3, 3);
			\coordinate (ta) at (7, 3);
			\coordinate (tb) at (7, 0);

			\draw (vx-up) to[R=5<\ohm>, v=$v_x$] (vx-down);
			\draw (r15-down) to[R=15<\ohm>] (r15-up);
			\draw (vx-up) to[R=10<\ohm>] (r10b);
			\draw (r10b) to[cvsource=$4v_x$] (r15-up);
			\draw (vx-down) to[short] (r15-down);
			\draw (r15-up) to[short, -o] (ta) node[right]{a};
			\draw (r15-down) to[short, -o] (tb) node[right]{b};
		\end{circuitikz}
		\caption{Esquema para o problema prático 4.10}
		\label{fig:pp4.10}
	\end{figure}
	
		A forma mais simples é excitar o circuito com uma fonte de tensão de \SI{1}{\volt}, ou uma fonte de corrente de \SI{1}{\ampere}. Utilizaremos uma fonte de tensão, juntamente com análise de malhas, conforme a figura \ref{fig:pp4.10.1}.
		
		
			\begin{figure}[H]
			\centering
			\begin{circuitikz}[american, thick]
				\coordinate (vx-down) at (0, 0);
				\coordinate (vx-up) at (0, 3);
				\coordinate (r15-up) at (6, 3);
				\coordinate (r15-down) at (6, 0);
				\coordinate (r10b) at (3, 3);
				\coordinate (ta) at (7, 3);
				\coordinate (tb) at (7, 0);
				
				\draw (vx-up) to[R=5<\ohm>, v=$v_x$] (vx-down);
				\draw (r15-down) node[ground]{} to[R=15<\ohm>] (r15-up);
				\draw (vx-up) to[R=10<\ohm>] (r10b);
				\draw (r10b) to[cvsource=$4v_x$] (r15-up);
				\draw (vx-down) to[short] (r15-down);
				\draw (r15-up) to[short, -o] (ta) node[right]{a};
				\draw (r15-down) to[short, -o] (tb) node[right]{b};
				\draw (tb) to[vsource, l_=$1V$, invert] (ta);
			\end{circuitikz}
			\caption{Circuito excitado pela fonte de tensão de \SI{1}{\volt}}
			\label{fig:pp4.10.1}
		\end{figure}
		
		Teremos duas correntes, $I_0$ e $I_1$ fluindo no sentido contrário pelas duas malhas. Elas se encontram e seguem o mesmo sentido no resistor de \SI{15}{\ohm}, em direção à referência (0 do circuito). É importante notar quê:
		
		$$
		R_{th} = \frac{v_a}{I_0} = \frac{1}{I_0}
		$$
		
		Escrevemos a primeira LKT, para a corrente de malha $I_1$, conforme a equação \ref{eq:7}. 
		
		\begin{equation}
			\centering
			\begin{aligned}
				-v_x + 10i_1 + 4vx +15(i_1 + i_0) &= 0 \\
				3v_x + 10_i + 15i_1 + 15i_0 &= 0 \\
				3v_x + 25i_1 + 15i_0 &= 0 \\
				v_x &= -5 \cdot i_1 \\
				3(-5i_1) + 25i_1 + 15i_0 &= 0 \\
				-15i_1 + 25i_1 + 15i_0 &= 0 \\
				15i_0 + 10i_1 &= 0
			\end{aligned}
			\label{eq:7}
		\end{equation}
		
		Iremos escrever a segunda LKT, para a corrente de malha $I_0$, conforme a equação \ref{eq:8}.
		
		\begin{equation}
			\begin{aligned}
				15(i_0 + i_1) - 1 &= 0 \\
				15i_0 + 15i_1 &= 1
			\end{aligned}
			\label{eq:8}
		\end{equation}
		
		Com as duas equações para as correntes de malhas $i_1$ e $i_0$, resolvemos o sistema de equações \ref{eq:9}.
		
		\begin{equation}
			\begin{cases}
				15i_0 + 10i_1 = 0 \\
				15i_0 + 15i_1 = 1
			\end{cases}
			\label{eq:9}
		\end{equation}
		
		Logo, com o valor de $i_0$, podemos calcular $R_{th} = \frac{1}{-0.133} \approx \SI{-7.52}{\ohm}$.

\end{document}