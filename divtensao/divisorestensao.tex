\documentclass[]{article}
\usepackage[siunitx, RPvoltages]{circuitikz}
\usepackage{amsmath}
\usepackage{graphicx}
\usepackage[utf8]{inputenc}
\usepackage[portuguese]{babel}
\usepackage{float}
\usepackage{imakeidx}
\usepackage{hyperref}

\makeindex

%opening
\title{Divisores de Tensão}
\author{Julio C B Gardona}
\date{\today}

\begin{document}

\maketitle
\clearpage
\tableofcontents
\clearpage

\begin{abstract}
	
Este texto tem por objetivo ser uma abordagem simples ao estudo de divisores de tensão. Contém exemplos de casos dos uso mais comuns, como divisores com 2 e 3 resistores, além de alguns exercícios.
\end{abstract}

\section{Introdução}
\index{Introdução}
A teoria do divisor de tensão é um conceito fundamental na eletrônica que \textbf{descreve como a tensão é distribuída entre resistores conectados em série} em um circuito. Em termos simples, um divisor de tensão permite obter uma tensão de saída menor a partir de uma tensão de entrada maior, usando apenas resistores.

\section{Conceito Básico}
\index{Conceito Básico}
Quando dois ou mais resistores são conectados em série em um circuito, a corrente que passa por eles é a mesma. No entanto, a tensão é dividida entre os resistores, de forma que a soma das tensões em cada resistor seja igual à tensão total aplicada ao circuito.

\subsection{Como Calcular um Divisor de Tensão?}\index{Como calcular um divisor de tensão?}

Pela lei de Ohm podemos definir a equação do divisor de tensão:

\begin{itemize}
	\item A resistência equivalente $T_t$ é:
	$$
	R_t = \sum_{x=1}^n R_x
	$$
	
	\item A tensão $V_x$ é:
	$$
	V_x = I_x \cdot R_x
	$$ 
	
	\item A corrente total no circuito é:
	$$
	I_t = I_x = \frac{V_t}{R_t}
	$$
	
	\item Manipulando a primeira equação, chegamos no \textit{divisor de tensão} abaixo:
	$$
	V_x = \frac{V_t \cdot R_x}{R_t}
	$$
	
	\item Para calcular a tensão $V_x$ em cima do resistor $R_x$ com $n$ resistores:
	$$
	V_x = V_t \cdot \left( \frac{R_x}{R_t} \right)
	$$
\end{itemize}

\section{Exemplo Prático}\index{Exemplo Prático}

Imagine um circuito com uma fonte de tensão de $V_t=\SI{12}{\volt}$ e dois resistores $R_1=\SI{1}{\kilo\ohm}$ e $R_2=\SI{2}{\kilo\ohm}$. Calcule $ V_1$ e $V_2$:

\begin{equation}
	\centering
	\begin{aligned}
		R_t &= R_1 + R_2 = 1000 + 2000 = \SI{3}{\kilo\ohm} \\
		V_1 & = V_t \cdot \frac{R_1}{R_t} = 12 \cdot \frac{1000}{3000} = \SI{4}{\volt} \\
		V_2 & = V_t \cdot \frac{R_2}{R_t} = 12 \cdot \frac{2000}{3000} = \SI{8}{\volt}
	\end{aligned}
\end{equation}

\section{Divisor de Tensão com 2 Resistores}\index{Divisor com 2 resistores}
Considere projetar um circuito com 2 resistores em série, onde a corrente seja de $I_t = \SI{0.02}{\ampere}$, a tensão $V_t = \SI{3.3}{\volt}$, e $V_2 = \SI{1.5}{\volt}$.

\begin{itemize}
	\item Podemos começar por $R_2$, pois possuímos $V_2$:
	
	$$
	R_2 = \frac{V_2}{I_t} = \frac{1.5}{0.02} = \SI{75}{\ohm}
	$$
	
	\item Para calcular $V_1$, podemos usar a \textbf{LKT, ou Lei de Kirchhoff das Tensões}:
	
	$$
	V_t - V_1 - V_2 = 3.3 - V_1 - 1.5 = \SI{1.8}{\volt}
	$$
	
	\item Para $R_1$, usaremos a \textit{Lei de Ohm}:
	
	$$
	R_1 = \frac{V_t}{I_t} = \frac{1.8}{0.02} = \SI{90}{\ohm}
	$$
\end{itemize}

\section{Divisor de Tensão com 3 Resistores}\index{Divisor com 3 resistores}

Considere projetar um circuito divisor de tensão, com três resistores, sendo o terceiro em paralelo com a saída de tensão. A corrente total do circuito $I_t=\SI{0.06}{\ampere}$, a tensão $V_t=\SI{3.3}{\volt}$. A carga a ser acoplada representa um \textit{LED vermelho}, que tem uma tensão direta $V_l=\SI{2}{\volt}$, e uma corrente de operação $I_l=\SI{0.03}{\ampere}$.

\begin{itemize}
	\item Começaremos por $R_l$, pois temos sua corrente e tensão:
	
	$$
	R_l = \frac{V_l}{I_l} = \frac{2}{0.03} = \SI{66.666}{\ohm}
	$$
	
	\item Para $R_2$:
	
	\begin{equation}
		\centering
		\begin{aligned}
			R_2 &= \frac{V_l}{I_t}=\frac{2}{0.06} = \SI{33.333}{\ohm} \\
			R_{2l} &= R_2 \parallel R_l = \frac{R_2 \cdot R_l}{R_2 + R_l} = \frac{33.333 \cdot 66.666}{33.333 + 66.666} = \SI{22.222}{\ohm}
		\end{aligned}
	\end{equation}
	
	\item Iremos calcular $R_1$. É necessário derivar a equação do divisor de tensão para encontrar seu valor:
	
	\begin{equation}
		\centering
		\begin{aligned}
			V_o &= V_t \cdot \frac{R_1}{R_{2l}} = \left\{2 = 3.3 \cdot \frac{R_1}{22.222}\right\} \\
			R_1 &= \frac{R_{2l} \cdot V_o}{V_t} = \frac{22.222 \cdot 2}{3.3} = \SI{13.467}{\ohm}
		\end{aligned}
	\end{equation}
\end{itemize}

\begin{figure}[H]
\centering
\begin{circuitikz}[american]
	\draw (0, 0) node[ground]{} to[vsource, v=\SI{3.3}{\volt}, f=\SI{0.06}{\ampere}] (3, 0);
	\draw (3, 0) to[R, -*, l2_=$R_1$ and \SI{13.467}{\ohm}] (3, -3);
	\draw (3, -3) to[R, v^=\SI{2}{\volt}, l2_=$R_l$ and \SI{66.666}{\ohm}] (6, -3) node[ground]{};
	\draw (3, -3) to[R, l2_=$R_2$ and \SI{33.333}{\ohm}] (3, -6) node[ground]{};
\end{circuitikz}
\caption{Resolução}
\label{fig:ex1}
\end{figure}

\section{Exercícios}\index{Exercícios}

\subsection{Exercício 1}
Para a configuração na figura \ref{fig:ex2}, responda:

\begin{itemize}
	\item Por inspeção visual, qual resistor receberá a maior porção da tensão aplicada?
	
	O resistor $R_3$.
	
	\item Quão maior será a tensão $V_3$, em comparação com $V_2$ e $V_1$?
	
	$V_3$ será 100x maior que $V_1$, e 10x maior que $V_2$. $V_3 = 10 \cdot V_2$ e $V_3 = 100 \cdot V_1$.
	
	\item Descubra a tensão do maior resistor usando a regra dos divisores de tensão.
	
	$$
	V_2 = V_t \cdot \frac{R_3}{R_t} = \SI{30}{\volt} \cdot \frac{20000}{22200} \approx \SI{27.027}{\volt}
	$$
	
	\item Descubra a tensão através de uma combinação em série dos resistores $R_2$ e $R_3$.
	
	$$
	V' = V_t \cdot \frac{R_2 + R_3}{R_t} = 30 \cdot \frac{20000 + 2000}{22200} \approx \SI{29.729}{\volt}
	$$
\end{itemize}

\begin{figure}[H]
	\centering
	\begin{circuitikz}[american]
		\draw (0, 0) node[ground]{} to[vsource, v=\SI{30}{\volt}] (2, 0);
		\draw (2, 0) to[R, l2=$R_1$ and \SI{200}{\ohm}] (2, -2);
		\draw (2, -2) to[R, l2=$R_2$ and \SI{2}{\kilo\ohm}] (2, -4);
		\draw (2, -4) to[R, l2=$R_3$ and \SI{20}{\kilo\ohm}] (2, -6) node[ground]{};
	\end{circuitikz}
	\caption{Circuito Exemplo}
	\label{fig:ex2}
\end{figure}

\subsection{Exercício 2}
\begin{itemize}
	\item Projete um divisor de tensão que permitirá o uso de uma lâmpada de \SI{9}{\volt} e \SI{60}{\milli\ampere}, onde o sistema elétrico é alimentado por uma fonte de \SI{12}{\volt}.
	
	\begin{equation}
		\centering
		\begin{aligned}
			R_1 &= \frac{V_t - V_l}{I_l} = \frac{12 - 9}{0.06} \approx \SI{50}{\ohm} \\
			P_1 &= V_1 \cdot I_t = \SI{180}{\milli\watt}
		\end{aligned}
	\end{equation}
	
	\item Qual é a especificação de potência mínima do resistor calculado se estão disponíveis resistores de $\frac{1}{8}W$, $\frac{1}{4}W$ e $\frac{1}{2}W$?
	
	Como a potência dissipada por $R_1$ é de $\SI{180}{\milli\watt}$, podemos utilizar um resistor de $\SI{250}{\milli\watt}$.
	
	\item Descubra a tensão através de cada resistor na figura \ref{fig:3}, se $R_1=2R_2$ e $R_2=7R_3$.
	
	\begin{equation}
		\begin{aligned}
			V_x &= V_t \cdot \frac{R_x}{R_t} \\
			R_t &= R_1+R_2+R_3 = 2R_2+7R_3+R_3 \\
			R_t &= 2 \cdot 7 \cdot R_3+7R_3+R_3 = 14R_3 + 8R_3 = 22R_3 \\
			V_3 &= V_t \cdot \frac{R_3}{R_t} = V_t \cdot \frac{R_3}{22R_3} = \frac{V_t}{22} = \frac{40}{22} = \SI{1.818}{\volt} \\
			V_2 &= 7 \cdot V_3 = 7 \cdot 1.818 = \SI{12.727}{\volt} \\
			V_1 &= 2 \cdot V_2 = 2 \cdot 12.727 = \SI{25.454}{\volt}
		\end{aligned}
	\end{equation}
	
	\begin{figure}[H]
		\centering
		\begin{circuitikz}[american]
			\draw (0, 0) node[ground]{} to[vsource, v=$\SI{40}{\volt}$] (0, 2.5);
			\draw (0, 2.5) to[R, v=$V_1$, l=$R_1$] (2.5, 2.5);
			\draw (2.5, 2.5) to[R, v=$V_2$, l=$R_2$] (2.5, 0);
			\draw (2.5, 0) to[R, v=$V_3$, l=$R_3$] (2.5, -2.5) node[ground]{};
		\end{circuitikz}
		\caption{Circuito exemplo}
		\label{fig:3}
	\end{figure}
	
	
\end{itemize}

\subsection{Exercício 3}

Projete um circuito divisor de tensão que permitirá uma tensão de \SI{5}{\volt} no resistor $R_2$ no circuito da figura \ref{fig:4}. Qual a especificação de potência mínima para os resistores calculados se estão disponíveis resistores de \SI{125}{\milli\watt}, \SI{250}{\milli\watt} e \SI{500}{\milli\watt}?

\begin{equation}
	\begin{aligned}
		R_2 &= \frac{V_2}{I_t} = \frac{5}{0.04} = \SI{125}{\ohm} \\
		V_1 &= V_t - V_2 = 12 - 5 = \SI{7}{\volt} \\
		R_1 &= \frac{V_1}{I_t} = \frac{7}{0.04} = \SI{175}{\ohm}
	\end{aligned}
\end{equation}

A especificação de potência mínima para $R_1$ é  de \SI{500}{\milli\watt} e para o resistor $R_2$ é de \SI{250}{\milli\watt}.

\begin{figure}[H]
	\centering
	\begin{circuitikz}[american]
		\draw (0, 0) node[ground]{} to[vsource, v=\SI{12}{\volt}] (0, 2);
		\draw (0, 2) to[short, f=\SI{40}{\milli\ampere}] (2, 2);
		\draw (2, 2) to[R=$R_1$] (2, 0);
		\draw (2, 0) to[R=$R_2$, v=\SI{5}{\volt}] (2, -2) node[ground]{};
	\end{circuitikz}
	\caption{Circuito exemplo}
	\label{fig:4}
\end{figure}

\printindex

\end{document}
